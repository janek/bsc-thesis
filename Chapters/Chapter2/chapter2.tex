%!TEX root = ../thesis.tex
%*******************************************************************************
%*********************************** Second Chapter *****************************
%*******************************************************************************

\chapter{Related work}  %"Related work and state of the art"?

\ifpdf
    \graphicspath{{Chapters/Chapter2/Figs/Raster/}{Chapters/Chapter2/Figs/PDF/}{Chapters/Chapter2/Figs/}}
\else
    \graphicspath{{Chapters/Chapter2/Figs/Vector/}{Chapters/Chapter2/Figs/}}
\fi

I present related work in three categories. The first one is an introduction to invertebrate tracking 
and automated observation, along with an overview of the state of the art. 

The second focuses on the division of labor in honeybees. It’s meant to set foundation for the analysis 
that we undertake in this work, as well as for determining what other kinds of approaches should be accessible 
given the BeesBook dataset and the building blocks that this work adds to it.

Finally, the last one collects works similar to this contribution - ones that use the \textit{BeesBook} 
dataset to perform some analysis of a honeybee colony’s life and/or add their own functionalities or improvements 
to the system.


%********************************** %First Section  **************************************
\section{Invertebrate observation and tracking} 

Observing invertebrates at scale, before a certain degree of automation was possible, required 
a lot of careful manual work and some creative approaches. A fascinating example of how experiments 
were conducted back then can be found in \citep{seeley_adaptive_1982}. The authors mark a hundred bees out of a 
colony of 21 thousand, using a brush with pigment mixed with shellac (following the example set 
by \citep{von_frisch_tanzsprache_1965}). They then pick for observation small subsections of the hive 
(quadrants), employing the help of a Texas Instruments calculator to generate randomness for their choices. 
Inferences about the entire population are made using the samples, but even to observe the samples, 8 hours 
of continuous work per day, for over 20 days, was necessary. To create maps of activity, authors used glass 
sheets that they put markings on and exchanged every day. They then photographed the sheets and projected the 
photographs against a single sheet of paper, one by one, thereby aggregating the one-day information sets 
into a single map. They also used a number of other physical and numerical tricks to 
be able to produce quality data. 


%********************************** %Second Section  **************************************



%********************************** %Third Section  **************************************



%********************************** %Fourth Section  **************************************
